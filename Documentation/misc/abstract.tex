\thispagestyle{empty}
\vspace*{1.0cm}

\begin{center}
    \textbf{Abstract}
\end{center}

\vspace*{0.5cm}

\noindent
The trend of developing chatbots at a corporate level has been raised for the last few years. The existing state of the art dialogue frameworks like IBM's Watson Assistant \cite{ibmwatson}, RASA \cite{rasa}, and PLATO \cite{plato} are providing great support in this regard. But still there exist various hindrances that need to be addressed in order to make this emerging technology noticeable. It is necessary to keep the frameworks as simple as possible without compromising on its abilities so that a non-technical person could also be able to develop a chatbot according to his/her needs.
\\~\\
This thesis illustrates the design, implementation, and evaluation of the chatbot framework developed using novel Modular Architecture. The chatbot designed using this framework is named as "Frankenbot". The chatbot uses the RASA's Natural Language Understanding(NLU) Model for the semantical interpretation of the users' utterances in order to fetch the required fields for the chatbot. 
\\~\\
Categorically, it provides the designers and developers to follow a simpler structure for creating a new chatbot with an additional added feature of modules re-usability. Furthermore, it has also provided support for users to chat on multiple topics at the same time. It means once the user switches the dialogue from one topic to another then instead of reinstating the chatbot, he/she can resume the chat for the previous topic where he/she left off.
\\~\\
This newly introduced strategy has been evaluated with the help of the research study conducted using different surveys. The users gave their opinion about what they experienced while interacting with the Frankenbot. They also provided a judgment for the quality of the chatbot. The majority of the users rated the overall experience with the chatbot as good. Also, the hedonic and pragmatic quality of the chatbot was graded as neutral but near to desirable.

 
 


