\chapter{Introduction\label{cha:chapter1}}
\epigraph{"You all know that chatbots are a new technology altogether. It’s like the early age of the Web. Things are still shaky yet growing at the speed of light."}{Rashid Khan (Build Better Chatbots)}
\\~\\
As manifested by Khan and Das (2018) {\textit{"chatbots are a new technology altogether"} \cite{buildBetterChatbots}} which depicts that future belongs to the chatbots. So to cope up with the speed of advancement, it is the need of the time to make chatbots intelligent enough to communicate like humans. However, many states of the art chatbots are developed but still, they have room for improvement. And for this purpose, a lot of research is happening around the globe.
\\~\\
Mostly, already established chatbots can deliver for what they are designed and perform tasks for humans. Chatbots like Google Assistant, Apple's Siri, and Amazon's Alexa are among the most developing virtual assistants and becoming the need of everyday life. Whereas, IBM's Watson Assistant provides a cloud service for developers to include the service in their software and design the chatbots according to their choice and needs. In addition to it, there are several existing developed open source frameworks used widely for this purpose. RASA framework is one of these open source structures to provide ease in performing machine learning stuff and natural language understanding(NLU). Such frameworks play an important role in developing intelligent chatbots in the present era. 
\\~\\
The prime purpose of this master's thesis is to develop a framework with a different approach i.e. modular approach as mentioned in Chapter 3 of this document. It assists developers to design their own conversational interfaces by using this framework. As an example, the chatbot named Frankenbot has been developed using this modular framework under the scope of this master's thesis requirements.
\\~\\
The fundamental goal of this study is to introduce a framework and make a contribution to the currently rising field of the chatbots. The advanced modular framework has been developed using Python. It implements a web API to make a request to the server via a user interface. And a JSON response with desired information is returned to display it for a user. Web API has been developed using Python's library known as Flask. The chatbot implemented using this framework is named as Frankenbot. Which has been used for experimental purposes and evaluating the research accomplished in this master's thesis. 
\\~\\
Frankenbot itself is a detective chatbot that communicates with users and on the basis of their respective answers makes a judgment whether a user is a culprit of a robbery or not.

\section{Motivation}
The concept of conversational agents knows as chatbots have been there for decades. But Brandtzaeg and Følstad (2018) highlighted the year 2016 as a revolutionary year for it. And real rationales behind it were rapid growth in the field of artificial intelligence(AI) and a rise in a trend for messenger applications such as Facebook Messenger and Slack etc. \cite{ChatbotsChangingUserNeedsMotivations}. Human-like conversational systems are one of  the  emerging  hot  topics  nowadays. Retrospectives and advancements in artificial intelligence (AI) and drastic transformation of mindset i.e. humans communicating with some automated agent without being recognized are the main reasons for enhancement of interest towards chatbots. It is very important to make sure that chatbots are intelligent enough to understand user utterances and the semantics in order to communicate as humans do.  The other important fact that can’t be deprecated is unlike living beings, machines don’t need refreshment and can perform 24/7.
\\~\\
The word chatbot is derived from “chat robot”. It  clearly  states  that  it  is  an  automated agent dealing with natural language user interfaces for data and services provided via dictation or writing. Users can query, command, or converse with the chatbots using regular language in order to get the required content in the form of data or service.  As one messaging platform provider Kik, claims on its developer site:  “First there were websites, then there were apps.  Now there are bots.” \cite{ChatbotsChangingUserNeedsMotivations}. And no one can deny the fact that future belongs to the bots as you can easily notice that many companies are cutting out the man force in their customer support sector and replacing it with "Chatbots". With more advancement in these conversational agents, they can be utilized for many other departments at a corporate level.
\\~\\
The existing state of the art dialogue frameworks like RASA \cite{rasa}, PLATO \cite{plato}, and IBM Watson \cite{ibmwatson} manage dialogues modeled as a single dialogue tree and contains only single state for all modules where module can be an utterance, response pair or a dialogue tree. It has several disadvantages like a complex tree structure, complex and lengthy transitions between nodes, difficult to keep track of dialogue, dependency on the last user utterance to stay in topic, and difficulty in switching and jumping to and fro between different topics. To overcome these issues there is a need for an invention that has been proposed by means of this master's thesis as explained below.

\section{Scope of the Thesis}
The goal of this master thesis is the development of a framework to rapidly develop conversational interfaces/chatbots.
\\~\\
% mentioned in \cite{modularFram}
To overcome the problems with already existing frameworks mentioned lastly, the framework is needed that follows and implements the modular architecture for conversational interfaces. Following new advancements has been added to this approach: (i) Dialogues are not modeled as a single dialogue tree but as multiple modules, (ii) A module can be a single utterance/response pair or a dialogue tree and (iii) The dialogue can have one state in each module instead of only a single state in all modules. These progressions exposed several benefits. Transitions between the dialogue states do not need to be modeled explicitly. Therefore dialogue trees can be simpler. Furthermore, the chatbot can develop a sense of “staying in the topic” instead of choosing the topic simply based on the last user utterance.

\subsection{Challenges}
This section highlights the major challenges to design a chatbot with modular dialogue manager. 
\begin{itemize}
  \item Recognition of NLU based intent and entity(s). Designing a tree structure for groups of intents, entities and assign parent and child node relation among them.
   \item Combining multiple trees based on user utterances to make sure that chatbot can stay in topic without changing its state to some new topic.
   \item Implementing a dialogue manager that favors recent modules. It means once the module is triggered, dialogue manager can save it as an active module. There are more chances of the next user utterance to belong to this active module which results in enhancing the performance.
   \item The modules should be able to create dialogue trees. Every node of
   the dialogue tree contains an answer followed by the JSON based file structure for storage.
   \item Implementing the web based frontend with a support of multiple user sessions at the same time, relies on a web API with a single endpoint which receives the user utterance as a parameter and returns the chatbot answer.
   \item Handling of Natural language processing by picking up the syntax. For example, If we query a chatbot "what's the weather?". It will respond perfectly fine but what if we rephrase it and ask "Could you please check the weather?" there is a chance of a glitch. These kind of programming problems resides under natural language processing category and are centre of attraction for the companies like Facebook, Google with Deep Text and Syntax Net respectively. \cite{ProgrammingchallengesofChatbot}
   \item When it comes to Machine Learning, it is another fact that can not be declined while designing and developing a chatbot. Efficient programming practices with artificial intelligence(AI) concepts is a key to achieve the best learning for a system to respond correctly \cite{ProgrammingchallengesofChatbot}.
   \item Choosing a pipeline for RASA's natural language understanding(NLU) to extract the best out of it.
\end{itemize}

\subsection{Contributions}
This document contains a review of the literature about dialogue managers in chatbots. Moreover, it also explains a software architecture and its implementation in addition to the necessary configuration files and the dialogue JSON file.
\\~\\
The following functionalities have been successfully implemented during the thesis:
\begin{itemize}
  \item Natural language understanding(NLU) for user utterance performed using RASA to detect intent and entities based on data for what it has been trained.
   \item Dialogue trees; The modules are capable of creating dialogue trees. Every node of the dialogue tree contains an answer.
   \item JSON based file structure to store the dialogues.
   \item The modular dialogue manager. Its detailed functionality and working have been discussed ahead in Chapter 3.
   \item The NLU based intent recognition has been implemented using the RASA NLU interpreter.
   \item The chatbot also performs RASA's NLU based entity recognition. All detected entities are written to the whiteboard.
   \item Dialogue manager is responsible to generate a response for a user utterance based upon the detected intent with the highest confidence. Secondly, if there is an entity key that appears in a bot response then it gets updated with its value stored on a whiteboard. On the contrary, it just gets eliminated in case of no value stored for that key.
   \item Frontend: A prototypical web-based frontend has been developed as an interface for a user. This web-based interface relies on a  web API with a single endpoint that receives the user utterance as a parameter and returns the chatbot answer.
   \item Multiple users can chat simultaneously using the web-based interface. Their dialogue states, whiteboards, and other session-based variables will not mix.
   \item The system generates a log file that helps to analyze why the dialogue manager comes to a certain answer for a user utterance and what information and results have been provided by RASA's natural language understanding(NLU).
%   \item Directory name containing JSON files for bot's structure, training data, and configuration information for RASA's NLU should be provided as a command-line parameter.
   \item To demonstrate the functionality of the framework a chatbot is implemented using the framework i.e. detective bot named Frankenbot.
\end{itemize}

\subsubsection*{Technical Requirements}
The framework has developed using Python language. It includes all the above-mentioned functionalities and is usable, which means it is running without any technical bug or error. Lastly, it has no dependencies on libraries with problematic licenses. Only the libraries lie under open source license have been utilized during the implementation of a framework.

\section{Structural Outline}
The current chapter provides a general introduction to the thesis topic. Whereas, Chapter 2 i.e. Foundations and Related Work, discusses the history of the chatbots along with the detailed background. Additionally, it also illustrates the classifications of dialogue systems. Furthermore, the medium of communication, chatbot elements along with their respective tasks and conversational agents applications have been discussed. And finally, the chapter gets closed with an explanation of evaluation techniques. 
\\~\\
Moreover, Chapter 3 explains the concept and design of the system. It elaborates on the system overview, architecture, and capabilities of a system. Also, the detailed description of the whole framework and dialogue manager developed in this thesis has been described in this chapter.
\\~\\
Next, Chapter 4 expresses the users' experience for a new approach. It also represents the results gathered by the completion of user surveys.
\\~\\
Lastly, Chapter 5 discusses and provides the analysis for the results of the evaluation portrayed previously in Chapter 4. Moreover, it also summarizes and concludes the work and tasks performed in this thesis. Finally, the chapter and this document get closed with an outlook for future work.