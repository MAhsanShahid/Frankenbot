\chapter{Discussion and Conclusion\label{cha:chapter5}}
This is the final chapter for this masters thesis which summarizes it. Along with that it also discusses the results of the evaluation illustrated in the Chapter \ref{cha:chapter4}. Lastly, the possible future work has been highlighted.

\section{Summary}
The motivation of this masters thesis was to implement a framework to enhance the development for conversational interfaces using a novel modular architecture discussed in the Chapter \ref{cha:chapter3} of the document. After its implementation, it has to be evaluated on the basis of the users experience and the quality. This has been accomplished by crafting and implementing the "Frankenbot", a modular architectural virtual conversational agent that communicates with the users and act as a detective to investigate about a robbery. This conversational agent then has been judged on the basis of the opinions collected by the users who played with it.
\\~\\
Chapter \ref{cha:chapter2} explains all the foundations and related work. Starting from the history and overview of the chatbots to their tasks and components. Furthermore, dialogue systems along with the existing state of the art frameworks have also been mentioned. In addition to that, Dialogue Management Systems(DMS) have been discussed in detail along with their challenges and evaluation methods.
\\~\\
In Chapter \ref{cha:chapter3}, the design and implementation of the modular chatbot(Frankenbot) has been explained in detail. The Frankenbot is made up of client(user interface) and a backend(web server) implemented in Python. And a client communicates with a server using Web API implemented using python's library named as Flask. For intent detection the framework for natural language understanding(NLU) named as RASA has been used after feeding and training it using a data source in JSON format. Th training data provided to it was just for the demo detective game along with few other topics like jokes, bot's profile and gossips etc. Web server receives the user utterance from the client using web API and does further processing accordingly which involves intent and entities detection using already trained Rasa's NLU Model. Once the intent has been detected the chatbot completes the remaining essential tasks granted to itself by its modular behaviour and finally produces a suitable response for the user. This response is sent back to the user and can be viewed on an interface that he/she has already used to send a request.
\\~\\
Chapter \ref{cha:chapter4} contains the research study that was developed and conducted to gather the results about the user's experience and the system's quality of the Frankenbot.
\\~\\
The practical results determined in the Chapter \ref{cha:chapter4} exposes that the users overall impression about the chatbot designed using modular architecture is good. Additionally, pragmatic and hedonic qualities along with the attractiveness of the system has been rated fair by the users but still there exists a room for improvement. Moreover, qualitative analysis highlights the problems with natural language understanding and interactivity. Also provides the feedback about new added feature, client and the detective game.

\section{Discussion\label{sec:discussion}}
There exists several state of the art frameworks for the chatbots but to the best of my knowledge none of them is following the novel modular architecture introduced during this research study. It has several advantages such as enhanced usability, simpler tree structure and connection between tree nodes, minimal redundancy as a module has to be declared only once. Other than that, it can act as a unified framework for different technologies. As a module can be more then a dialogue tree. Other systems (question answering, neural systems etc.) can also get fit in this framework as long as they implement an activation function. And there exist different methods to implement an activation function over the modules. 
\begin{enumerate}
    \item Each module calculates its activation independently based on the added modules in the chatbot.
    \begin{itemize}
        \item Pro: Easy and liberal implementation of modules.
        \item Con: Comparing the confidence values of different modules can lead to errors.
    \end{itemize}
    \item All modules use the same intent recognition module. Therefore the system can calculate a joint intent recognition over all modules.
    \begin{itemize}
        \item Pro: Better results are expected.
        \item Con: Module specific NLU is not possible. Combinations of different NLU strategies is limited.
    \end{itemize}
    \item NLU Chain: First an option 1 NLU is applied. If there is no match then the NLU of option 2 will be used.
\end{enumerate} 
Another advantage of using the framework following the modular architecture is that each module manages its own state. Therefore, it is straightforward to switch between modules and come back to the last state. These were some important technical facts that needed to be addressed here.
\\~\\
Coming back to the evaluation results and stats, the users and participants of the surveys liked the chatbot as majority of the respondents rated its overall impression as good. Also the AttrakDiff study has shown that this approach is practical and initial rating about the system's quality collected according to the users perspective is also worthwhile. Users also appreciated the feature of parallel topic handling and coming back and forth to different topics at same time without re-initiating the chatbot. There can be multiple modules within a same chatbot and each module maintains a separate state for each user which makes this feature possible.
\\~\\
Contrarily, the experiments and evaluation also exposed the deficiencies in this initially designed and implemented framework. These problems can be categorized as the technical problems as they occurred due to misbehaviour of the system and didn't meet the users expectations. Secondly, some conceptual problems have also been noticed.
\\~\\
Most common technical problem occurred was the useless or irrelevant response generated by the framework to the user utterance. And that happened just because of wrongly detected intent by the NLU. And the NLU caused this problem just because of limited resources and data available for training. Another problem encountered was again associated with NLU as if user types any meaningless word then the NLU still provides some result in form of the detected intent. Lastly, one conceptual problem was noticed by the users that the chatbot was not responding with meaningful answer if the user utterance is a long sentence with some additional information that is not required at that moment in a conversation. But sometimes the chatbot responded correctly as well. Again this problem raised due to the misleading caused due to an incorrect recognized intent by the NLU. But with the continuous usage, users shortened their inputs by providing only necessary terms required by the chatbot and then the NLU started working perfectly fine.

\section{Future Work}
The results have evidently demonstrated that the modular architecture implemented in the Frankenbot got success and rated comprehensively good by the users. But it still needs some optimization to enhance the user experience and pragmatic and hedonic quality of the system in order to reach a desirable mark for it. The deficiencies encountered for the chatbot have consigned in this section.
\\~\\
To boost the user experience and the abilities of the Frankenbot, it is an essential need to improve the training for natural language understanding. And it could be done using efficient and populous data set. This adjustment could be helpful for the advancement of natural language understanding for any type of the user utterances and statements imposed to the chatbot. Secondly, it can also be enhanced by applying live training using the data collected during the dialogue between the chatbot and the user. But it can lead to miscellaneous results. So, to make it work a developer should be responsible to check whether the detected intent for the user input was correctly identified and then added to the training data. Additionally, Rasa framework for nlu has been utilized for this purpose in the current version of the chatbot. But if there will be any other better and more advanced framework developed in the future then it can also be replaced for better understanding of the semantics for the user utterances.
\\~\\
For now, activation for the modules could be calculated using two of the following approaches independently: (i) each module calculates its activation independently based on the added modules in the chatbot. (ii) All modules use the same intent recognition module. Therefore the system can calculate a joint intent recognition over all modules. But for better results for the intent recognition the hybrid method "NLU Chain" could also be implemented as stated in the section \ref{sec:discussion}. 
\\~\\
Moreover, the chatbot has been designed using dialogue tree structure for its simplicity. It means a module should be designed in the form of a dialogue tree. Due to which it undergoes few limitations that a node can only be a child of only one different node. In other words, one node can only has one parent node assigned to it. Due to this limitation one module can not be appointed as a child to two different modules in different chatbots. To make it work, the dialogue graphs must be implemented instead of the dialogue trees. As the graph data structure supports the multiple parents feature which is not allowed in a tree. But as a reminder, the main purpose of the modular architecture was also to keep the framework simple to minimize the transitions between the nodes as discussed in the section \ref{par:simplerTree}. By using graphical structure it could become more complex.
\\~\\
Lastly, the response has been generated only on the basis of the detected intent in the current version of the chatbot. Its performance can also be upgraded by adding a feature of response generation using identified entities along with the recognized intent for the user utterance. By adding the support for this characteristic the chatbot could be able to produce more precised and relevant responses for the users statements.