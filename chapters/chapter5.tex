\chapter{Discussion\label{cha:chapter5}}
% This thesis presented and magnified the importance of hybrid approaches for multi-step ahead forecast. Furthermore, we have analyzed the different combining strategies based on the relationship between the linear and non-linear components of the time series. We saw that time series may consist of various components that cannot be modeled efficiently with a single model alone and it becomes more difficult to estimate the future point as the forecast horizon increases. We saw that the problem of multi-step ahead forecast is still an open question in the research field. 
% \\~\\
% In Chapter 2, we have briefly discussed the basic time series modeling techniques and the research that has been done to effectively model the time series. Likewise, we presented the hybrid approaches for single-step ahead forecast that assumed different relationships between the linear and non-linear components of the time series. We saw that the hybrid modeling techniques performed better for a single-step ahead forecast. We have also highlighted the basic non-linear multi-step forecasting techniques with their strengths and weaknesses.
% \\~\\
% In Chapter 3, we introduced the different hybrid approaches for multi-step ahead forecast. The new hybrid approaches used the assumption that the time series consists of linear and non-linear components. Three different combining strategies are used based on the relationship between the components. The linear hybrid approach for multi-step ahead considered the simple aggregate relationship between the linear and non-linear components as used by Zhang \cite{Zhang} for a single step ahead prediction. Furthermore, the aggregation relationship has been used with different multi-step ahead forecasting techniques. The generalized hybrid approach did not limit the relationship to be linear only, it utilized the non-linear model for the non-linear component as well as to combine the different components of the time series. The generalized hybrid approach has been used with different multi-step ahead forecasting techniques as well. Another hybrid approach that separates the modeling of non-linear component and combining the different patterns of time series has been used and evaluated with multi-step ahead forecasting techniques and called a non-linear hybrid approach. 
% \\~\\
% In Chapter 4, for the evaluation of the presented hybrid approaches different linear and non-linear models were selected and discussed in detail. The ARIMA and Seasonal ARIMA were selected as the linear models and ANN with different multi-step ahead forecasting techniques were used as non-linear models. Two different real-life time series were briefly discussed and used in evaluating the hybrid approaches with various forecast horizons. The implementation details as well as the possible values of hyperparameters were presented. The grid search method was used to select the values of hyperparameters of the models. Moreover, we highlighted the accuracy measurements which were used in evaluating the multi-step forecast. The hybrid approaches have been evaluated on both time series using a single feature and the results are presented in detail. Furthermore, the cross-validation method for time series has been used to determine the accuracy of the hybrid models on different test sets of both time series. The effect of additional features of both time series on the hybrid approaches is also highlighted.

% \\~\\
% Overall, the accuracy of the hybrid approach for a multi-step ahead forecast has been analyzed and compared with the individual linear and non-linear models. We saw that the hybrid models improved the accuracy of both time series for the multi-step forecast. However, it is not always true that every hybrid approach would result in improved accuracy. The accuracy of the hybrid approaches are subjective to the time series under consideration and depends on the model selection as well as the technique for multi-step forecast. Furthermore, we have tried to use the hybrid approaches for the multi-step forecasting on a large dataset of Power time series and despite running the experiment for two days, it could not complete because of the iterative process, i.e., the training of the linear model after calculation of every residual value. Based on the results presented in Chapter \ref{cha:chapter4}, the division of the time series into linear and non-linear components is an efficient hybrid technique but it is not very salable and cannot handle large time series. Finally, the best accuracy for multi-step forecast has been achieved by using the hybrid modeling. According to the results presented in this thesis the main observed variable of the time series is enough to achieve high prediction accuracy for multi-step ahead forecast.


% %It depends on various factors like time series components, correlation between different observations, the relationship between linear and non-linear patterns of the time series, the type of models used in hybrid approach and the technique for multi-step ahead forecast. % 

% %There are no proper guidelines available which models to combine in the hybrid approach, it is all based on exhausted trial-and-error approach, just like designing neural network for predicting problem. Furthermore, another limitation of using %



% % This approach was first used by Khashei and Bijari  \cite{Khashei:2011:NHA:1930543.1930730} for single step ahead forecast.%





% \section{Proper Hybrid Approach Selection\label{sec:futurework}}
% According to the experiment results, the hybrid approaches cannot always improve the accuracy and provide a better multi-step forecast. It depends on various factors like time series components, the correlation between different observations, the relationship between linear and non-linear patterns of the time series, the type of models used in hybrid approach and the technique for multi-step ahead forecast. 
% \\~\\
% Considering the experiment results, the model's selection plays an important role in the hybrid approaches to perform better and produce an accurate multi-step forecast. Nonetheless, there are no proper guidelines available about which models to combine in the hybrid approach, it is all based on trial-and-error method, just like designing a neural network for the forecasting problem. Another important factor is to decide the multi-step techniques for the non-linear models. If there is no autocorrelation between the forecast point of the time series then the Direct technique would provide a better result because it considers every future point as a separate and independent point. The direct technique is computationally intensive and directly related to the forecast horizon because it would create multiple models equal to the forecast horizon. If dependencies are present between forecast points then the Recursive or MIMO technique could be considered. The Recursive technique may suffer from the accumulation of the errors as the forecast horizon increases. On the other hand, the MIMO model should be generalized enough to understand the relationship within forecast points and between inputs and outputs.
% \\~\\
% The relationship between the linear and non-linear components can vary depending on the time series. If the relationship is linear and no dependency within the forecast points exist, then the Hybrid Linear Direct approach would be a suitable option otherwise, Hybrid Linear Recursive or Hybrid Linear MIMO approaches would be considered. If the relationship between the linear and non-linear patterns is not linear then the Hybrid Non-Linear approach can be more effective. Overall, it is possible to improve the forecast with the hybrid approach, if the right fitting hybrid model is applied.  
% \\~\\

% \section{Future Work\label{sec:futurework}}
% The results of the experiments clearly show that the accuracy of the multi-step forecasting can be improved by the proposed hybrid approaches. However, it is still an open problem. The results and insights of this thesis provide further open directions for multi-step forecasting with hybrid modeling. The most important challenge in the field of multi-step forecasting with hybrid modeling is the need to find the relationships between different components of the time series effectively. It would drastically reduce the trial-and-error method to find the right combining strategy. Another important challenge in the field of hybrid modeling for multi-step forecasting is the selection of the linear and non-linear models. The same selection for different time series cannot ensure the improvement in the accuracy of the hybrid approach for multi-step forecasting. There is a need for comprehensive guidelines for the selection of models that can be used in the hybrid approach. One of the challenging tasks which needs further research is to have extensive guidelines to design the architecture of the neural network for forecasting problem. This is considered to be the major bottleneck for the extensive use of ANNs in time series forecasting. 
% \\~\\
% Another open challenge in the field of hybrid modeling is to define a more appropriate technique that can be used for a large number of historical observations of the time series. The residual technique in hybrid modeling is not very scalable due to the iterative strategy for calculating the residual series, that is used for training of the non-linear model. An option for the efficient hybrid approach might be possible by dividing the training set for the linear and non-linear models and then combine them using another non-linear model without calculating the residual series. 
% \\~\\
% Furthermore, it would be interesting to investigate different models especially machine learning models including Support Vector Regression (SVR), Recurrent Neural Networks, and Gaussian Processes (GP) in hybrid modeling to see the effect on prediction accuracy for multi-step forecasting. Recently, Long short-term memory (LSTM) in recurrent neural networks have been spotted for time series forecasting and considered to be very effective where long term dependency exists between the forecast points. We have used three basic multi-step techniques, i.e., Recursive, Direct, and MIMO in hybrid approaches, it would be worthwhile to see other hybrid multi-step techniques for hybrid approaches like DIRMO, and their effect on the prediction accuracy. 
% \\~\\
% Finally, the field of hybrid modeling for time series forecasting lacks the ability to provide prediction intervals around the forecast point. Prediction intervals provide a range of possible values in which the future point may lie with a certain probability. The prediction intervals play a very important role in the field of time series forecasting. Especially with regard to multi-step forecasting, due to the increase in uncertainty and the chances to make errors in point forecasting. Indeed, the decision-makers can benefit immensely from the upper and lower prediction intervals. Linear models like ARIMA calculate the prediction interval with point estimation. The calculations are done based on Gaussian/normal distribution of the prediction errors. When using a hybrid model instead, error series does not imply normal distribution, so the same method cannot be used. It is an open challenge to calculate the prediction interval in the field of hybrid modeling for multi-step forecasting. One option to calculate the prediction interval with hybrid modeling is using re-sampling or bootstrap method \cite{898089,5966350}.

% %Even though the multiple features did not improve much the multi-step forecast, it still needs further investigation especially which features to include in
% %

% %The most important issue for further research in the field of hybrid approaches for multi-step forecasting is the need to find the right combination of models in hybrid approach. Another important challenge in the field of forecasting with hybrid modeling is the need to find the relationships between different components of the time series. It would drastically reduce the exhausting trial-and-error grid search method.