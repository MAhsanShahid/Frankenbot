\chapter{Discussion and Conclusion\label{cha:chapter5}}
This is the final chapter for this master's thesis which summarizes it. Additionally, it also discusses the limitations of the framework stated in Chapter \ref{cha:chapter3} and evaluation methodologies along with the gathered results illustrated in Chapter \ref{cha:chapter4}. Lastly, possible future work has been manifested.

\section{Summary}
The motivation of this master's thesis was to implement a framework to enhance the development of conversational interfaces using a novel modular architecture discussed in Chapter \ref{cha:chapter3} of the document. After its implementation, it has to be evaluated based on the users' experience and quality. This has been accomplished by crafting and implementing the "Frankenbot", a modular architectural virtual conversational agent that communicates with the users and acts as a detective to investigate a robbery. This conversational agent then has been judged based on the opinions collected by the users who played with it.
\\~\\
Chapter \ref{cha:chapter2} explains all the foundations and related work. Starting from the history and overview of the chatbots to their tasks and components. Furthermore, dialogue systems along with the existing state of the art frameworks have also been mentioned. In addition to that, Dialogue Management Systems(DMS) have been discussed in detail along with their challenges and evaluation methods.
\\~\\
In Chapter \ref{cha:chapter3}, the design, and implementation of the modular chatbot(Frankenbot) have been explained in detail. The Frankenbot is made up of a client(user interface) and a backend(webserver) implemented in Python. And a client communicates with a server using Web API implemented using python's library named as Flask. For intent and entities detection, the framework for natural language understanding(NLU) named as RASA has been used after feeding and training it using a data source in JSON format. The training data provided to it was just for the demo detective game along with few other topics like humor, bot's profile, and gossips, etc. The web server receives the user utterance from the client using web API and does further processing accordingly which involves intent and entities detection using already trained Rasa's NLU Model. Once the intent has been detected the chatbot completes the remaining essential tasks granted to itself by its modular behavior and finally produces a suitable response for the user. This response is sent back to the user and can be viewed on an interface that he/she has already used to send a request.
\\~\\
Chapter \ref{cha:chapter4} contains the research study that was developed and conducted to gather the results about the user's experience and the system's quality of the Frankenbot.
\\~\\
The practical results determined in Chapter \ref{cha:chapter4} exposes that the users' overall impression about the chatbot designed using modular architecture is good. Additionally, pragmatic and hedonic qualities along with the attractiveness of the system have been rated fair by the users but still there exists room for improvement. Moreover, the qualitative analysis highlights the users' experience and problems with the framework. It also provides feedback about the framework's new added feature, client, and the detective game.

\section{Discussion\label{sec:discussion}}
This section formally discusses the limitations and positive aspects of the implemented framework and approaches used for its evaluation. 

\subsection{Framework}
There exist several states of the art frameworks for the chatbots but to the best of my knowledge, none of them is following the modular architecture introduced during this research study. It has several advantages such as enhanced usability, simpler tree structure, and connection between tree nodes, minimal redundancy as a module has to be declared only once. Other than that, it can act as a unified framework for different technologies. As a module can be more than a dialogue tree. As a theory, other systems (question answering, neural systems, etc.) can also get fit in this framework as long as they implement an activation function. And there exist different methods to implement an activation function over the modules. 
\begin{enumerate}
    \item Each module calculates its activation independently based on the added modules in the chatbot.
    \begin{itemize}
        \item Pro: Easy and liberal implementation of modules.
        \item Con: Comparing the confidence values of different modules can lead to errors.
    \end{itemize}
    \item All modules use the same intent recognition module. Therefore the system can calculate a joint intent recognition overall modules.
    \begin{itemize}
        \item Pro: Better results are expected.
        \item Con: Module-specific NLU is not possible. Combinations of different NLU strategies is limited.
    \end{itemize}
    \item NLU Chain: First an option 1 NLU is applied. If there is no match then the NLU of option 2 will be used.
\end{enumerate} 
Another advantage of using the framework following the modular architecture is that each module manages its state. Therefore, it is straightforward to switch between modules and come back to the last state. These were some important technical facts that need to be addressed here.
\\~\\
Along with the positive aspects of it, several limitations have been encountered during its implementation. For intent and entity recognition, the RASA NLU component has been used. It appeared to be a challenging task as currently it has been fed with limited related data only as shown in Appendix \ref{appen:traindatastats}. But it needs a large data set for training purposes to perform well. Secondly, dialogue designing needs well-structured JSON data to be processed further by a dialogue manager. Moreover, the chatbot has been designed using a dialogue tree structure for its simplicity. It means a module should be designed in the form of a dialogue tree. Due to which it undergoes few constraints that a node can only be a child of only one different node. In other words, one node can only have one parent node assigned to it. Due to this limitation, one module can not be appointed as a child to two different modules within the same or different chatbots. Furthermore, large memory storage will be required to save the current state of each user for all the activated modules within complex dialogues. Depending upon the complexity of dialogue it may take longer to detect an intent as it has to process all the available modules separately. The selected intent would be the one with the highest confidence value. Lastly, only intent specific responses are currently available for the users.

\subsection{Evaluation}
The evaluation has been accomplished with the help of following methodologies: (i) Frankenbot's Experience Survey (Appendix \ref{appen:expsurvey}), (ii) Frankenbot's Evaluation via AttrakDiff (Appendix \ref{appen:attrsurvey}) and (iii) Interview (Section \ref{subsec:interview}). Several limitations have been faced during the evaluation process. Firstly, only 20 participants were allowed to fill the AttrakDiff's survey by default and this limit has been set by tool designers for using AttrakDiff's Single Evaluation Technique. For the other survey as it has been designed using Google Sheet and there was no such constraint on the number of participants. But to maintain the same number of participants, both of the surveys have been shared with the same 20 participants. While only 12 out of them have appeared for an interview. Secondly, due to the current pandemic, it was not possible to have an in-person meeting with the participants. So, for this reason, they have been contacted using social media platforms. It was also the cause of hindrance in the process. Additionally, the chatbot has been evaluated containing only two modules designed for demo Frankenbot i.e. detective and general as shown in Appendix \ref{appen:traindatastats}. And it has been judged for the user experience, communication, usability, design, qualities(Hedonic and Pragmatic), and framework's modular state handling. But it should also be evaluated and compared with any existing state of the art dialogue framework for determining its true potential.
\\~\\
Heading towards the collected evaluation results and stats, the users and participants of the surveys liked the chatbot as a majority of the respondents rated its overall impression as good. Also, AttrakDiff's study has shown that this approach is practical and initial rating about the system's quality collected according to the users' perspective is also worthwhile. Users also appreciated the feature of parallel topic handling and coming back and forth to different topics at the same time without re-initiating the chatbot. There can be multiple modules within the same chatbot and each module maintains a separate state for each user which makes this feature successful.
\\~\\
Contrarily, the experiments and evaluation also exposed the deficiencies in this initially designed and implemented framework. These problems can be categorized as technical problems as they occurred due to the misbehavior of the system and didn't meet the users' expectations. Secondly, some conceptual problems have also been noticed.
\\~\\
A common technical problem occurred was the useless or irrelevant response generated by the framework to the user utterance. And that happened just because of wrongly detected intent by the NLU. And the NLU caused this problem just because of limited resources and data available for training as shown in Appendix \ref{appen:traindatastats}. Another problem encountered was again associated with NLU as if user types any meaningless word then the NLU still provides some result in the form of the detected intent. Lastly, one conceptual problem was noticed by the users that the chatbot was not responding with a meaningful answer if the user utterance is a long sentence with some additional information that is not required at that moment in a conversation. But sometimes the chatbot responded correctly as well. This problem raised due to the misleading caused by incorrectly recognized intent by the NLU. But with the continuous usage, users shortened their inputs by providing only necessary terms required by the chatbot and the NLU worked as expected. This indicates that the system needs to be improved for natural and humanly conversation.
\\~\\
As it is just a simple demo chatbot designed under various limitations of time and resources but still able to capture positiveness out of the users. So, if the framework will undergo proper training and design then it might be a revolutionary transformation for the chatbots.

\section{Future Work}
The results have demonstrated that the modular architecture implemented in the Frankenbot got successful and rated comprehensively good by the users. But it still needs some optimization to enhance the user experience and pragmatic and hedonic qualities of the system to reach a desirable mark for it. The solutions for deficiencies listed in Section \ref{sec:discussion} are consigned here.
\\~\\
To boost the user experience and the abilities of the Frankenbot, it is an essential need to improve the training for natural language understanding. And it could be done using the efficient and populous data set. This adjustment could be helpful for the advancement of natural language understanding for any type of user utterances and statements imposed on the chatbot. Secondly, it can also be enhanced by applying live training using the data collected during the dialogue between the chatbot and the user. But it can lead to miscellaneous results. So, to make it work a developer should be responsible to check whether the detected intent for the user input was correct and then add it to the training data. Additionally, the Rasa framework for NLU has been utilized for this purpose in the current version of the chatbot. But if there will be any other better and more advanced framework developed in the future then it can also be replaced for a better understanding of the semantics for the user utterances.
\\~\\
Activation for the modules could be calculated using two of the following approaches independently: (i) each module calculates its activation independently based on the added modules in the chatbot. (ii) All modules use the same intent recognition module. Therefore the system can calculate a joint intent recognition overall modules. But for better results for intent recognition the hybrid method "NLU Chain" could also be implemented as stated in Section \ref{sec:discussion}. 
\\~\\
The dialogue graphs could also be implemented instead of dialogue trees. As the graph data structure supports the multiple parents' feature which is not allowed in a tree. But as a reminder, the main purpose of the modular architecture was also to keep the framework simple to minimize the transitions between the nodes as discussed in section \ref{par:simplerTree}. By using graphical structure it could become more complex.
\\~\\
The performance can also be raised by adding a feature of response generation using identified entities along with the recognized intent for the user utterance. By adding the support for this characteristic the chatbot could be able to produce more precise and relevant responses for the users' statements.
\\~\\
To measure the real potential and capabilities of the framework implemented in this master's thesis, it would be great if it gets compared with existing advanced frameworks like IBM's Watson Assistant, etc. It could be done by designing the same dialogue using Frankenbot's framework and existing dialogue frameworks. Afterward, the dialogue design and structure should be compared to check the simplicity and module usability for both of them. Also, the same user utterance should be inputted to both of them to observe differences between the results. Secondly, evaluation from the users' perspective is always a great idea. One could also just design the same dialogue using a modular dialogue manager and any other state of the art dialogue manager. Once it has been done, the evaluation strategies already mentioned in this study can also be used for the assessment of both of the dialogue frameworks.