\thispagestyle{empty}
\vspace*{1.0cm}

\begin{center}
    \textbf{Abstract}
\end{center}

\vspace*{0.5cm}

\noindent
% The concept of conversational agents knows as chatbots have been there for decades but the revolutionary year for it was 2016.  Human-like conversational systems are one of  the  emerging  hot  topics  now  a  days.   Retrospectives  and  advancements  in  artificial intelligence (AI) and drastic transformation of mindset i.e.  humans communicating with some automated agent without being recognized are the main reasons for enhancement of interest towards chatbots.  It is very important to make sure that chatbots are intelligent enough to understand user utterances and the semantics in order to communicate as humans do.  The other important fact that can’t be deprecated is unlike living beings, machines don’t need refreshment and can perform 24/7.
% \\~\\
% The  word  chatbot  is  derived  from  “chat  robot”.   It  clearly  states  that  it  is  an  automated agent dealing with natural language user interfaces for data and services provided via dictation or writing.  Users can query, command or conversate with the chatbots using regular language in order to get the required content in form of data or service.  As one messaging platform provider Kik, claims on its developer site:  “First there were websites, then there were apps.  Now there are bots.” \cite{ChatbotsChangingUserNeedsMotivations}. And no one can deny the fact that future belongs to the bots as you can easily notice that many companies are cutting out the man force in their customer support sector and replacing it with "Chatbots". With more advancement in these conversational agents, they can be utilized for many other departments in a company.
% \\~\\
% Existing state of the art dialogue frameworks like RASA \cite{rasa}, PLATO \cite{plato} and IBM Watson \cite{ibmwatson} manage dialogues modelled as a single dialogue tree and contains only single state for all modules where module can be an utterance, response pair or a dialogue tree. But what if a novel dialogue manager is introduced, the “modular dialog manager” having dialogues modelled as a multiple modules instead of single dialogue tree. The dialogue can have one state in each module instead of only a single state in all modules. This exposes several benefits, transitions between the dialogue states do not need to be modelled explicitly. Therefore, dialogue trees can be simpler. Furthermore, the chatbot can develop a sense for “staying in topic” instead of choosing the topic simply based on the last user utterance.
 
 


