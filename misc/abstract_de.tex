\thispagestyle{empty}
\vspace*{0.2cm}

\begin{center}
    \textbf{Zusammenfassung}
\end{center}

\vspace*{0.2cm}

\noindent 
Der Trend zur Entwicklung der Chatbots auf Unternehmensebene hat sich seit einigen Jahren verstärkt. Bestehende hochmoderne Dialog-Frameworks wie IBM's Watson Assistant \cite{ibmwatson}, RASA \cite{rasa} und PLATO \cite{plato} bieten diesbezüglich hervorragende Unterstützung. Es gibt jedoch noch verschiedene Hindernisse, die angegangen werden müssen, um diese aufkommende Technologie spürbar zu machen. Es ist notwendig, die Frameworks so einfach wie möglich zu halten, ohne ihre Fähigkeiten zu beeinträchtigen, damit eine nicht technische Person auch einen Chatbot nach ihren Bedürfnissen entwickeln kann.
\\~\\
Diese Arbeit veranschaulicht das Design, die Implementierung und die Evaluierung des Chatbot-Frameworks, das unter Verwendung einer neuartigen modularen Architektur entwickelt wurde. Der mit diesem Framework entworfene Chatbot heißt "Frankenbot". Der Chatbot verwendet das NLU-Modell (Natural Language Understanding) der RASA zur semantischen Interpretation der Äußerungen des Benutzers, um die erforderlichen Felder für den Chatbot abzurufen.
\\~\\
Grundsätzlich können Designer und Entwickler die einfachere Struktur zum Erstellen eines neuen Chatbots mit zusätzlichen Funktionen zur Wiederverwendbarkeit von Modulen befolgen. Darüber hinaus wurde den Benutzern die Möglichkeit geboten, gleichzeitig über mehrere Themen zu chatten. Sobald der Benutzer den Dialog von einem Thema zum anderen wechselt, kann er den Chat für das vorherige Thema fortsetzen, in dem er aufgehört hat, anstatt den Chatbot wiederherzustellen.
\\~\\
Diese neu eingeführte Strategie wurde mit Hilfe der Forschungsstudie bewertet, die anhand verschiedener Umfragen durchgeführt wurde. Die Benutzer gaben ihre Meinung zu dem ab, was sie bei der Interaktion mit dem Frankenbot erlebt haben. Sie beurteilten auch die Qualität des Chatbots. Die Mehrheit der Benutzer bewertete die Gesamterfahrung mit dem Chatbot als gut. Auch die hedonische und pragmatische Qualität des Chatbots wurde als neutral, aber nahezu wünschenswert eingestuft.