% \thispagestyle{empty}
% \vspace*{0.2cm}

% \begin{center}
%     \textbf{Zusammenfassung}
% \end{center}

% \vspace*{0.2cm}

% \noindent 
% Prognosen spielen in verschiedenen Lebensbereichen eine wichtige Rolle, insbesondere in den Bereichen Business Intelligence, Meteorologie und Finanzen. Für bessere strategische Entscheidungen, Markttrends und Planungen sind genaue mehrstufige Prognosen von großer Bedeutung. Der Bereich der Zeitreihenanalyse wird seit langem von statistischen Modellen dominiert. In den letzten Jahren haben maschinelle Lernmethoden jedoch an Bedeutung gewonnen und sich als vielversprechende Alternative zu klassischen statistischen Modellen im Bereich der Zeitreihenanalyse erwiesen. Sie haben die Fähigkeit, die zugrundeliegende stochastische Abhängigkeit zwischen historischen und zukünftigen Beobachtungen zu erlernen. Es wurde beobachtet, dass jede der Modellierungstechniken nur für bestimmte Zeitreihenmuster geeignet ist. Daher werden besonders die hybriden Ansätze in der Community berücksichtigt. Die Aufteilung der Zeitreihen in lineare und nichtlineare Komponenten wird als sehr effiziente Hybridtechnik angesehen und wird in der Literatur überwiegend verwendet. Sowohl theoretische als auch empirische Erkenntnisse zeigen, dass die Hybridmodellierung besonders effektiv für die Verbesserung der Vorhersageleistung ist. Die hybriden Ansätze werden jedoch weitgehend nur in Bezug auf Einzelschrittvoraussagen verwendet. In dieser Arbeit werden verschiedene Techniken des maschinellen Lernens für mehrstufige Prognosen analysiert und in neu vorgeschlagenen Hybridansätzen zur weiteren Verbesserung der Prognosegenauigkeit verwendet. Die vorgeschlagenen Hybridansätze kombinieren lineare und nichtlineare Modellierungstechniken. Diese Arbeit zeigt, dass die Genauigkeit der mehrstufigen Vorhersage durch Hybridmodellierung verbessert werden kann. Darüber hinaus wird die Auswirkung mehrerer Merkmale auf die Genauigkeit der Vorhersagen unter Verwendung hybrider Ansätze für die mehrstufige Vorhersage analysiert. Schließlich werden die vorgeschlagenen hybriden mehrstufigen Ansätze analysiert und mit den vorhandenen mehrstufigen Modellen verglichen.